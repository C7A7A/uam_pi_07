\addcontentsline{toc}{chapter}{Zakończenie}
\chapter*{Zakończenie}
Praca przedstawiła podstawowe koncepty związane z tworzeniem REST API. Na początku pokazany i opisany został protokół HTTP oraz jego rozszerzenie - HTTPS. W pracy zawarte zostały podstawowe informacje na temat metod HTTP oraz popularnych kodów odpowiedzi, które należy wykorzystywać w architekturze REST. W kolejnej części praca skupiła się na istocie REST API, a więc samej architekturze REST oraz zasadach, które należy przestrzegać, by określić serwis RESTowym. Rozdział drugi, „Tworzenie REST API” położył nacisk na praktyczne aspekty programowania interfejsu sieciowego. Rozpoczął się od sekcji dobrych praktyk, gdzie opisany został szereg istotnych elementów przy tworzeniu REST API, od dokumentacji i wersjonowania oprogramowania po obsługę błędów oraz testowanie. Następnie praca przedstawiła dwa mechanizmy, które pozwalają na implementację uwierzytelniania użytkowników, opisała, czym jest OAuth oraz wytłumaczyła na czym polega autoryzacja. W ostatniej sekcji zademonstrowany został plan, który warto znać podczas projektowania REST API.

W dobie ciągłego przepływu informacji w sieci Internet należy tworzyć REST API, które są przyjemne w użytkowaniu oraz zdatne do utrzymania w celu zmaksymalizowania swojej atrakcyjności dla potencjalnych klientów. Współcześnie, wykrystalizowały się pewnie branżowe standardy, których warto trzymać się przy implementacji REST API.

W tej pracy wytłumaczono, czym właściwie jest REST API oraz przedstawiono wiele praktyk, które warto wdrożyć podczas programowania interfejsów sieciowych, aby dostarczyć użytkownikom oprogramowania produkt wysokiej jakości..